\documentclass{article}

\usepackage[utf8]{inputenc}
\usepackage{amsmath, amssymb}
\usepackage{hyperref}
\usepackage{cancel}
\usepackage{tikz}
\usepackage{setspace}  
\usepackage[a4paper, margin=2cm]{geometry}  

\onehalfspacing 

\setlength{\parskip}{0pt}
\setlength{\parindent}{0pt}

\setlength{\baselineskip}{0pt}
\hypersetup{
    colorlinks=true,
    linkcolor=black,
    filecolor=black,
    urlcolor=black,
}

\title{Math Notes}
\author{by me}
\date{\today}

\begin{document}

\maketitle

This is a work in progress.

$$\lim_{me \rightarrow insanity} me$$

\tableofcontents

\section{Arithmetic}
\subsection*{Order of Operation}
1. Parenthesis

2. Exponents

3. Multiplication \& Division (left to right)

4. Addition \& Substraction

\subsection{Basics}
$\forall a,b,c \in \mathbb{R} $ gilt:

Kommutativgesetz:

$a+b = b+a \text{ und } a*b=b*a$

Assoziativgesetz:

$(a+b)+c = a+(b+c)$ und $(a*b)*c=a*(b*c)$
 
Distributivgesetz

$a*(b+x) = a*b+a*c$ und $(a+b)*c = a*c + b*c$

\subsection{Fractions}
Rules:
\begin{align*}
    \frac{a}{b} + \frac{c}{b} &= \frac{a+c}{b} \\
    \frac{a}{b} - \frac{c}{b} &= \frac{a-c}{b} \\
    \frac{a}{b} \cdot \frac{c}{d} &= \frac{ac}{bd} \\
    \frac{a}{b} \div \frac{c}{d} &= \frac{a}{b} \cdot \frac{d}{c} = \frac{ad}{bc} \\
    \frac{a}{\frac{b}{c}} &= \frac{a \cdot c}{b} = \frac{ac}{b} \\
    \frac{1}{\frac{b}{c}} &= \frac{c}{b}
\end{align*}

\subsection{Exponents}
Rules:
\begin{align*}
    a^n \cdot b^n = (ab)^n \\
    a^n \cdot a^m = a^{n+m} \\
	(a^n)^m = a^{nm}=a^{mn}=(a^{m})^{n} \\ 
	\frac{a^n}{b^n}=(\frac{a}{b})^n \\ 
    \frac{a^n}{b^n}=a^{n-m}
\end{align*}

\subsection{Roots}
Rules:
\begin{align*}
    \sqrt[n]{a} \cdot \sqrt[n]{b} = \sqrt[n]{ab} \\ 
    \sqrt[n]{\sqrt[m]{a}} = \sqrt[m]{\sqrt[n]{a}} = \sqrt[nm]{a} \\
    \frac{1}{\sqrt[n]{a}} = \sqrt[n]{\frac{1}{a}} (a \neq 0) \\ 
    \frac{\sqrt[n]{a}}{\sqrt[n]{b}} = \sqrt[n]{\frac{a}{b}} (b \neq 0) \\ 
    (\sqrt[n]{a})^m = \sqrt[n]{a^m}
\end{align*}

Wichtig!

$\sqrt{a+b} \neq \sqrt{a} + \sqrt{b}$
$\sqrt{x^2} \neq x $

\subsection{Logarithms}

$$ \log_a{y} = x \Leftrightarrow a^x = y $$

\begin{align}
    \log_a(xy) = \log_a{x} + \log_a(y) \\ 
    \log_a(\frac{x}{y})=\log_a(x) - log_a(y) \\ 
    \log_a(x^r) = r \log_a(x) \\
    \log_a(x) = \frac{\log_b(x)}{\log_b(a)}
\end{align}
Auf (3) folgt:
$\log(\sqrt{x}) = \log_a(x^{\frac{1}{2}})=\frac{1}{2} \log_a(x)$


\subsection{Binomial Equations}
\begin{align*}
    \text{B1: }(a+b)^2 = a^2 +2ab +b^2 \\ 
    \text{B2: }(a-b)^2 = a^2 -2ab +b^2 \\
    \text{B3: }(a+b)(a-b) = a^2 - b^2
\end{align*}


\section{Linear Equations}
$f(x) = mx+b, \quad m,b \in \mathbb{R} $

Steigung:

Gegeben sind: Zwei Punkte $P(x_1|y_2) \text{ und } Q(x_2|y_2)$

$m := \frac{\Delta y}{\Delta x}:=\frac{y_1 - y_2}{x_1 - x_2} (= \frac{y_2-y_1}{x_2-x_1})$

$b :=$ verschiebung auf der Y Achse

parralel $(f \parallel g)$, falls $m_1 = m_2$

orthogonal $ (f \perp g) $ oder senkrecht, falls $m_1 \cdot m_2 = -1$

\subsection{Berechnung der Schnittpunkte von $f: \mathbb{R} \rightarrow \mathbb{R}, f(x)=mx+b$ mit den Achsen}

y-Achse:

$f(0) = m \cdot 0 + b = b$, (0|b) ist der schnittpunkt mit der y-Achse

x-Achse:

löse diese Gleichung:

$f(x) = 0  \Longleftrightarrow mx + b = 0 \Longleftrightarrow mx = -b $

Fallunterscheidung:

\[
x =
\begin{cases} 
-\frac{b}{m}, & \text{if } m \neq 0 \text{ (one unique x-intercept)} \\
\mathbb{R}, & \text{if } m = 0 \text{ and } b = 0 \text{ (entire x-axis is the solution)} \\
\text{no solution}, & \text{if } m = 0 \text{ and } b \neq 0 \text{ (no x-intercept)}
\end{cases}
\]

\fbox{Die x-Koordinate eines Schnittpunktes mit der x-Achse heißt Nullstelle der Funktion}
 % Seite 41

\section{Quadratic Equations}
We use quadratic equations for a bunch of shit, for example modelling of
procesesses.

\subsection{General form of an quadratic equation}

$$f(x)=ax^2 + bx +c \quad \text{ with } a,b_1,c_1 \in \mathbb{R}, a \neq 0$$

geschrieben in absteigender Reihenfolge der Potenzen

Standard Parabola:
$f(x)=x^2$

Example:
$f(x)=2x^2+6x+4$

\subsection{Scheitelpunktform}

$f(x)=a(x-b_1)^2+c_2 \quad a,b_1,c_1 \in \mathbb{R}, a \neq 0$
wobei \( (b_1, c_1) \) der Scheitelpunkt ist


Wenn a=1 ist nennt man die Funktionsgleichung in beiden Formen normiert.

Example:

$f(x)=2(x-2)^2+1$

$\Rightarrow V(2, 1)$

\subsection{Factored Form}

$f(x)=a(x-r_1)(x-r_2)...(x-r_n)$

Where $r_n$ are the Zero Points

Example:
$f(x)=-3(x+1)(x+3)$, if "x" is positive then it's the negative zero point
so it's -1 and -3, I think

\subsection{Mitternachtsformel und p - q Formel}

$$x_1,2 = \frac{-b \pm \sqrt{b^2 - 4ac}}2a{}$$

\[
D =
\begin{cases} 
\text{Zwei Lösungen}\in\mathbb{R}, & \text{if } D=b^2 - 4ac > 0 \\
\text{genau eine Lösung: } x = \frac{-b}{2a}, & \text{if } D=b^2 - 4ac =0 \\
\text{Keine Lösung in} \mathbb{R}, & \text{if } D=b^2 - 4ac < 0 
\end{cases}
\]




\section{Symbolic Notation}
$\forall$ - for all / every / any

$\exists$ - there exists

$\nexists$ - there does not exists

$\iff$ - if and only if

$\le$ - less then

$\leq$ - less then or equal to

$\ge$ - greater then

$\geq$ - greater then or equal to

$\Box$ - end of proof

$:=$ - Zuweisung

\subsection*{Examples}

b:= battery  

p:=power

$B=\{set\ of\ batteries\}$

$B_L=\{live\ batteries\}$

$B_d=\{dead\ batteries\}$

If $ b \in B_L $, then there is p

$ b \in B_L \Rightarrow \exists p $

If b is in a set of live batteries, there is power

$ b \in B_d \Rightarrow \nexists p $

b is a subset of $B_d$, if and only if there does not exist power

$\nexists \Rightarrow b \in B_d$

if there is no power, the battery is in the set of dead batteries

$b \in B_L$

$\therefore \exists p$  (this means therefore exists p)

b is in $B_L$ (live batteries), therefore there must be power.

\section{Sets}
\subsection{Set Builder Notation}

Example: Infinite set of even Integers

$E=\{..., -6, -4, -2, 0, 2, 4, 6, ...\}$

Because we cannot write out the entire Set we can use Set Builder Notation
to represent it.
Like so:

$E=\{2n : n \in \mathbb{Z}$

Often times uppercase letters are used to stand for sets this is not
a rule it just makes things more readable.

Some special sets or sets with significance are denoted using special uppercase letters
like $\mathbb{Z},\mathbb{N},\mathbb{Q}$.

Sets aren't ordered, meaning that $\{0,2,4,6\}$ = $\{2,6,4,0\}$

\subsection{Cardinality}

Cardinality is an attribute of sets, it's the number of elements a set
posesses, i.e. $A={0,2,4,6}$; $|A|=4$, the set A posseses 4 elements, meaning
that it's cardinality is that of 4.

We denote cardinality using these bar symbols: $| |$
because we use the same for absolute values you need to be careful to
not confuse the two. Again the absolute value of a number would be something
like: $|-4|=4$.

Here's an example where we use the absolute value in set builder notation:

$B=\{ x \in \mathbb{Z}: |x|<4 \}=\{-3,-2,-1,0,1,2,3\}$

Why are there negative numbers here? Well we are looking for numbers
that have an absolute value that is smaller then 4 and -3 has an absolute
value of 3 and that is indeed smaller than 4. As you see we are not
doing an operation in set builder notation but "filtering" values, you could
say.

\section{Linear Algebra}

\subsection{Overview}

\subsection{The Geometry of Linear Equations}

The Fundamental Problem of linear algebra: Solving a system of linear equations

\subsubsection*{Example}

What's the coefficient matrix?

a matrix is just a rectengular array of numbers

a coefficient is a numerical factor in front of a variable in a term.

A term is a expression or a part of an expression seperated by + or - signs.

For example in 2x+5, you got two terms: 2x and 5.

We have n equations and n unknows,
so we have the normal/nice case of equal number of equations and unknowns

Two equations, two unknowns.

$2x-y=0$
and

$-x+2y=3$

This gives us the matrix picture:

$$
\overset{A}{\begin{bmatrix}
 2 \- 1 \\ 
 -1\ 2
\end{bmatrix}}
\overset{x}{\begin{bmatrix}
x \\ y 
\end{bmatrix}}
= \overset{b}{\begin{bmatrix}
0 \\ 3
\end{bmatrix}}
$$





In this one we have two rows and two columns
\subsubsection*{Picture 1: Row picture}
First we describe the row picture

\subsubsection*{Picture 2: Column picture}
Using a Matrix we'll call A

\subsection{COPIED FROM OLD ORG FILE: Introduction}
A way to solve many mathematical problems is to translate/ reduce them into linear algebra.
After that the problems in linear algebra reduces down to the solving of a system of linear equatuion,
which in turn comes down to the manipulation of matrices.

So it's:

mathematical problem  $\rightarrow$ linear algebra $\rightarrow$ linear equations $\rightarrow$ matrix manipulation


Linear Algebra's power doesen't just lie in the manipulation of matricies and the solving
of linear equations, but also in that it allows us to abstract *concrete objects* down into the
ideas of vector spaces and linear transformation, allowing us to link many different concepts
together. Or just reveal these links between seemingly very different topics.

Not only linear algebra has this power, as any good abstraction, meaning some mathematical
system like linear algebra, has this power.

*there are many ways of telling when a system of $n$ linear equations in $n$ unknowns have a solution.*


\subsection{ The Basic Vector Space: $R^n$}

The quintessential vector space is $R^n$, the set of all n-tuples of real numbers

$\{ \left( x_{1},\ldots ,x_{n}\right):x_i\in\mathbb{R}\}$

$x_i$ is a n-tuple, meaning a tuple of somekind of, or rather any size, $x_i$ is part of the set R.

$R^n$ is a set of all n-tuples.

you can also look at this like that:


$a=\begin{pmatrix} x_1 \\ \vdots \\x_n \end{pmatrix}, a \in \mathbb{R}$

at least I think you can. Not too sure!


you can  add together two n-tuples to get another n-tuple:
$(x_1,\ldots, x_n)+(y_1,\ldots, y_n)=(x_1+y_1,\ldots, x_n+y_n)$

another way to present this would be:

$$\begin{pmatrix} x_1 \\ \vdots \\ x_n \end{pmatrix} + \begin{pmatrix} y_1 \\ \vdots \\ y_n \end{pmatrix} = \begin{pmatrix} x_1+y_1 \\ \vdots \\ x_n+y_n \end{pmatrix}$$

again, no idea if this is legitimate.

We can also multiply each n-tuple by some real number $\lambda$, here lambda just is a skalar, and just represents some real number.

$\lambda(x_1, \ldots , x_n)=(\lambda \cdot x_1, \ldots, \lambda \cdot x_n)$

represented by a column vector:

$\lambda \cdot \begin{pmatrix} x_1 \\ \vdots \\ x_n \end{pmatrix} = \begin{pmatrix} \lambda \cdot x_1 \\ \vdots \\\lambda \cdot x_2 \end{pmatrix}$

this is also known as multiplying by a skalar, the skalar is simpy a real number.

We call these n-tuples vectors and the real numbers $\lambda$ are called scalars.


The natural map or morphisim from some $R^n$ to and $R^m$ is given by matrix

\subsection{What are Natural Maps?}

Natural Maps are connection between different mathematical objects that arise naturally.
They arise from them without using any additional mathematical object of function, meaning
the map comes from the properties of the thingie itself. The natural map preserves the function
of the mathematical objects.


Now let's do a morphism between $R^n$ and $R^m$

Let's write a vector in $R^m$ as a column vector with $m$ entries

Let $A$ be an m $\times$ n matrix



$$
A=\begin{pmatrix} a_{11} & a_{12} & \ldots & a_{1n} \\ \vdots & \vdots & \vdots & \vdots \\ a_{m1} & \ldots & \ldots & a_{mn} \end{pmatrix}
$$


Then $Ax$ is the m-tuple:

$$
Ax=\begin{pmatrix} a_{11} & a_{12} & \ldots & a_{1n} \\ \vdots & \vdots & \vdots & \vdots \\ a_{m1} & \ldots & \ldots & a_{mn} \end{pmatrix} \begin{pmatrix} x_1 \\ \vdots \\ x_n \end{pmatrix} = \begin{pmatrix} a_{11} \cdot x_1 + \ldots + a_1n \cdot x_n \\ \vdots \\ a_{m1} \cdot x_1 + \ldots + \ a_mn \cdot x_n \end{pmatrix}
$$

For any two vectors $x$ and $y$ in $R^n$ and any two scalars $\lambda$ and $\mu$, we have:

$$
A(\lambda x+\mu y)=\lambda Ax+\mu Ay
$$

Dieser Ausdruck ist einfach nur dafür da um die gesamten operationen von $R^n$ und $R^m$ zu zeigen

$x,y$ sind vektoren, $A$ ist eine Matrix, $\lambda$ ist ein Skalar, was der Ausdruck zeigt ist wie Skalare,
Vektoren und Matricen miteinander interagieren, jedenfalls ist das mein read.

Now we relate what we have learned to the solving of a system of linear equations

System of linear equations:

$$
\begin{aligned}
    a_{11}x_{1} + \ldots + a_{1n}x_{n} & = b_{1} \\
    & \hspace{0.14cm} \vdots \\
    a_{m1}x_{1} + \ldots + a_{mn}x_{n} & = b_{m}
    \end{aligned}
$$

If we have a few equations, that's fine, but if we have a
fuckton, it become a nightmare to work with them,
so we can just rewrite the problem like that:
$$
b=\begin{pmatrix} b1 \\ \vdots \\ b_{m} \end{pmatrix}, A=\begin{pmatrix} a_{11} & a_{12} & \ldots & a_{1n} \\ \vdots & \vdots & \vdots & \vdots \\ a_{m1} & \ldots & \ldots & a_{mn} \end{pmatrix}
$$
and we write our unknowns as:

$$x=\begin{pmatrix} x_1 \\ \vdots \\x_n \end{pmatrix}$$

This way we can rewrite our system of linear equation in this
form:

$Ax = b$

$m = equations$

$n = unknows$

when $m > n = no\ solutions$

when $m < n = many\ solutions$


\section{Calculus}
\subsection{Basics}
In Calculus we mostly have derivatives and integrals. The relationship between the two is
reciprocal, more on that later.

Let's start with derivatives.

\subsection{Limits}
Limits are all about the behaviour of vairables/ functions when approaching
a number or infinity.

They are denoted by $ \lim_{x \rightarrow X} $
where $X$ is the number they are approaching.
Here's an example:

$\lim_{x \rightarrow \pm \infty} 2x ^2 + x^2 = + \infty$

Here we took the limit of the equation $2x^2 + x^2$ and determined that the
equation approaches positive infinity.

Basically how we do this is we put in larger and larger numbers to see how the
function behaves, if the function keeps growing it approaches positive infinity,
if it keeps decreasing it is approaching negative infinity.

We can of course also take limits where we approach a variable to a real number

like for example:

$\lim_{x \rightarrow \pm -2} 2x ^2$

As we approach $-2$ from BOTH directions, we can look at what solution the function approaches,
like we put in -1.9999999 or -2.1111111 we can see that the function keeps approaching the
number 8.


This stuff always gave me problems, because I know mathematics as a sort of precises thing
and limits always seemed unprecise, but hey they work and yes they are that simple. It threw
me off too at first.


\subsection{Derivatives}
When we want to know how much a function changes at a point, we take the derivative and
plug in the point in the new function we get.


Point: p
\begin{align}
f(x)=x^2
\\
f'(x)=2x
\\
p=2
\\
f'(2)=2*2=4
\end{align}

Now what do we do when we derive a function?

We basically look at how much a function changes at every point and encode this information
into a function.

There are derivative rules you can use to get there, but to fully understand
whats going on we should go through the steps manually, at least once,
but a few times would be better.

\begin{align*}
    \frac{df(h)}{dh} & = \lim_{\Delta \rightarrow 0} \frac{\Delta f}{\Delta h} = \lim_{\Delta h \rightarrow 0} \frac{f(h + \Delta h) - f(h)}{\Delta h}  
    \\
    f(h) &= h^2 & & \quad  \text{add } \Delta h:
    \\
    f(h + \Delta h) &= (h + \Delta h)^2 & & \quad \text{solve Parenthesis}
    \\
    f(h + \Delta h) &= h^2 + 2h\Delta h + \Delta h^2 & & \quad \text{subtract } f(h)
    \\
    f(h + \Delta h) - f(h) &= h^2 + 2h\Delta h + \Delta h^2 - h^2 & & \quad \text{cancel out } h^2
    \\
    f(h + \Delta h) - f(h) &= 2h\Delta h + \Delta h^2 & & \quad \div \Delta h
    \\
    \frac{f(h + \Delta h) - f(h)}{\Delta h} &= \frac{2h\Delta h + \Delta h^2}{\Delta h} & & \quad \text{factor out } \Delta h
    \\
    \frac{f(h + \Delta h) - f(h)}{\Delta h} &= \frac{\cancel{\Delta h}(2h + \Delta h)}{\cancel{\Delta h}} & & \quad
    \\
    \frac{f(h + \Delta h) - f(h)}{\Delta h} &= 2h + \Delta h & & \quad \text{now  take the limit: } \lim_{\Delta h \to 0}
    \\
    f'(h) &= \lim_{\Delta h \to 0} \frac{f(h + \Delta h) - f(h)}{\Delta h} = 2h & & \quad
\end{align*}


\subsection{Derivative Rules}
Now of course we don't always want to do all that to take the derivative.
That's why we use some rules.

\subsubsection{Power Rule}
\begin{align*}
    \frac{d}{dx}(x^n)=n \cdot x^{n-1}
    \\
    \frac{d}{dx}[x]=1
    \\
    \frac{d}{dx}[i]=0
    \\
    i \text{ here is some number}
\end{align*}

Example

\begin{align*}
    f(x)=x^3
    \\
    \frac{d}{dx}=3 \cdot x^2
\end{align*}

\subsubsection{Product Rule}
\begin{align*}
    \frac{d}{dx}[f(x) \cdot g(x)] = f(x) \cdot g'(x) + g(x) \cdot f'(x)
\end{align*}

\subsubsection{Quotient Rule}
\begin{align*}
    \frac{d}{dx}[\frac{f(x)}{g(x)}] = \frac{g(x)f'(x)-f(x)g'(x)}{[g(x)]^2} \ [if\ g(x) \neq 0]
\end{align*}

\subsection{Integrals}
\subsubsection{Trigonometric Substitution}
The universal trig sub

\begin{align*}
    tan(\frac{x}{2}) \leftarrow\rightarrow t
    \\\\
    sin(x) \leftarrow\rightarrow \frac{2t}{1+t^2}
    \\\\
    cos(x) \leftarrow\rightarrow \frac{1-t^2}{1+t^2}
    \\\\
    dx \leftarrow\rightarrow \frac{2dt}{1+t^2}
\end{align*}

\subsubsection*{Example 1}
$\int \underbrace{csc(x)}_{\text{cosecant}} dx $
the standard way to solve this would be something
like this:

$\int csc(x) dx \cdot \underbrace{\frac{csc(x)-cot(x)}{csc(x)-cot(x)}}_{\text{standrad way of solving this}}$

but we could also just use the substitution table.

$\int csc(x)dx \stackrel{\text{use table}}{=} \frac{1}{\frac{2t}{1+t^2} \cdot \frac{2dt}{1+t^2}}$

$\int \frac{\cancel{1+t^2}}{2t} \cdot \frac{\cancel{2}dt}{\cancel{1+t^2}} = \int \frac{dt}{t}$

$\int \frac{dt}{t} =ln |t| +C $ use table

$\int \frac{dt}{t} = ln |tan(\frac{x}{2})+X|$

the solution doesn't look the same when doing it the standard way, but it is the same!

\subsubsection*{Example 2}

$\int \frac{dx}{2+cos(x)}=\int \frac{1}{2+\frac{1-t^2}{1+t^2}} \cdot \frac{2dt}{1+t^2}$

$\Rightarrow \int \frac{2dt}{2(1+t^2)+1-t^2} \Rightarrow \int \frac{2dt}{2+2^t+1-t^2}$

$\Rightarrow \int \frac{2dt}{3+t^2}$

you can take the two out of the top because it's just a skalar mutliple

$2\int \frac{dt}{3+t^2} = \int \underbrace{\frac{du}{1+u^2}}_{\text{this is just the arctan of u}}$

the trick here is to do the following substitution:
\begin{align*}
    t=u\sqrt[2]{3}
    \\\\
    dt=\sqrt[2]{3}du
    \\\\
    \Rightarrow \frac{1}{\sqrt[2]{3}}dt = du
\end{align*}

$\frac{2}{\sqrt{3}}arctan(u)+C$

$\frac{2}{\sqrt{3}}arctan(\frac{1}{\sqrt{3}}tan(\frac{X}{2}))+C$

\subsection{Kurvendiskussion Example}
Unsere Funktion: $f(x)=2x^2+x^4$

\subsubsection{Definitionsbereich}

Definintionsbereich bestimmen basically in welchem Zahlenraum sich die Funktion bewegt
hier ist es.

$D_f=\mathbb{R}$

\subsubsection{Grenzwerte, Verhalten im undendlichen}

\section{Duale Zahlen}

\end{document}
